\section{Data transmission}
\subsection{Types and methods of data transmission}
\begin{point}
	\begin{enumerate}[label=(\alph*)]
		\setlength\itemsep{0em}
		\item Understand that data is broken down into
			packets to be transmitted
		\item Describe the structure of a packet
		\item Describe the process of packet switching
	\end{enumerate}
\end{point}

A data packet consists of the following:
\begin{itemize}
	\item \emph{\ul{Packet header}}: Consisting of three 
		further pieces of information:
		\begin{itemize}
			\item \emph{\ul{Destination address}}: Address
				of the recipient of the packet.
			\item \emph{\ul{Packet number}}: A number used 
				reshuffle the packets back into sequence.
			\item \emph{\ul{Originator's address}}: Address
				of the sender.
		\end{itemize}
	\item \emph{\ul{Payload}}: Consisting of the actual 
		data contents of the packet.
	\item \emph{\ul{Trailer}}: Consists of some data
		required in error checking, and a signal to
		indicate the ending of the packet.
\end{itemize}

\begin{point}
	\begin{enumerate}[label=(\alph*)]
		\setlength\itemsep{0em}
		\item Describe how data is transmitted from one
			device to another using different methods
			of data transmission
		\item Explain the suitability of each method of
			data transmission, for a given scenario
	\end{enumerate}
\end{point}

The methods of data transmission can be largely separated
into two, based on amount of data transmitted and
direction of data being transmitted.
Based on amount of data transmission:
\begin{itemize}
	\item \ul{\emph{Serial}}: Data is transmitted \emph{one
		bit	at a time, down a single wire.} Such 
		transmission is time consuming, but safer in that
		data will always arrive in order and has less chance
		of being skewed as, due to the fewer amount of wires,
		there is less chance of interference. The usage of a
		single wire also makes this method relatively less
		expensive.
	\item \ul{\emph{Parallel}}: Data is transmitted \emph{
		multiple bits at a time, down multiple wires.} Here,
		due to there being multiple wires and each of them transmitting
		at different speeds, data does not always arrive in order and may
		even be skewed due to interference across long distances. Yet, 
		the fact that data travels parallelly, makes the transmission faster. However,
		the usage of multiple wires makes parallel transmission expensive.
		Internally, a computer transmits in parallel, so transmitting in parallel
		requires no extra processing for conversion.
\end{itemize}
Based on direction of data transmission:
\begin{itemize}
	\item \ul{\emph{Simplex}}: Data is transmitted in 
		\emph{only one direction.}
	\item \ul{\emph{Half duplex}}: Data is transmitted in
		\emph{both directions, but not at the same time.}
	\item \ul{\emph{Duplex}}: Data is transmitted in 
		\emph{both directions at the same time.}
\end{itemize}

\begin{point}
Understand the universal serial bus (USB)
interface and explain how it is used to transmit
data
\end{point}

The \emph{\ul{universal serial bus (USB)}} is an industry standard of data transmission.
Flash drives, mouse connectors are all subsets of this interface. Advantages of the USB
interface follow:

\begin{itemize}
	\item A USB cable fits only one way, meaning no wrong connections can be made.
	\item USB transmission is relatively high, so data can be transferred quickly.
	\item Since USB is an industry standard, a USB port is included on almost all
		devices.
	\item USB insertion is always automatically detected, which begins the installation
		of required drivers.
	\item USB connections can also be used to power a device, which can be used to
		charge devices (mobile phones).
\end{itemize}
Conversely, the disadvantages of the interface are:
\begin{itemize}
	\item The length of a USB cable is always limited, upto 5 metres.
	\item The transmission speed for USb, though it is high, it is not as high
		ethernet.
\end{itemize}

\subsection{Methods of error detection}
\begin{point}
Understand the need to check for errors after
data transmission and how these errors can occur
\end{point}

During transmission, due to factors such as interference, errors may arise in the data
transmitted, such as loss of data, gain of data and change of data. Using data with
errors will cause problems, hence data must be checked for errors.

\begin{point}
Describe the processes involved in each of the
following error detection methods for detecting
errors in data after transmission: parity check
(odd and even), checksum and echo check
\end{point}

