\section{Data transmission}
\subsection{Types and methods of data transmission}
\begin{point}
	\begin{enumerate}[label=(\alph*)]
		\setlength\itemsep{0em}
		\item Understand that data is broken down into
			packets to be transmitted
		\item Describe the structure of a packet
		\item Describe the process of packet switching
	\end{enumerate}
\end{point}

A data packet consists of the following:
\begin{itemize}
	\item \emph{\ul{Packet header}}: Consisting of three 
		further pieces of information:
		\begin{itemize}
			\item \emph{\ul{Destination address}}: Address
				of the recipient of the packet.
			\item \emph{\ul{Packet number}}: A number used 
				reshuffle the packets back into sequence.
			\item \emph{\ul{Originator's address}}: Address
				of the sender.
		\end{itemize}
	\item \emph{\ul{Payload}}: Consisting of the actual 
		data contents of the packet.
	\item \emph{\ul{Trailer}}: Consists of some data
		required in error checking, and a signal to
		indicate the ending of the packet.
\end{itemize}

\begin{point}
	\begin{enumerate}[label=(\alph*)]
		\setlength\itemsep{0em}
		\item Describe how data is transmitted from one
			device to another using different methods
			of data transmission
		\item Explain the suitability of each method of
			data transmission, for a given scenario
	\end{enumerate}
\end{point}

The methods of data transmission can be largely separated
into two, based on amount of data transmitted and
direction of data being transmitted.
Based on amount of data transmission:
\begin{itemize}
	\item \ul{\emph{Serial}}: Data is transmitted \emph{one
		bit	at a time, down a single wire.} Such 
		transmission is time consuming, but safer in that
		data will always arrive in order and has less chance
		of being skewed as, due to the fewer amount of wires,
		there is less chance of interference.
	\item \ul{\emph{Parallel}}: Data is transmitted \emph{
		multiple bits at a time, down multiple wires.}
\end{itemize}
Based on direction of data transmission:
\begin{itemize}
	\item \ul{\emph{Simplex}}: Data is transmitted in 
		\emph{only one direction.}
	\item \ul{\emph{Half duplex}}: Data is transmitted in
		\emph{both directions, but not at the same time.}
	\item \ul{\emph{Duplex}}: Data is transmitted in 
		\emph{both directions at the same time.}
\end{itemize}
