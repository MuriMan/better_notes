\section{Software}
\subsection{Types of software and interrupts}

\textbf{(1)}\textit{ Describe the difference between system software
and application software and provide examples
of each.}

System software provides the services that a computer requires, which includes the operating
system and utility software. Application software provides services required by the user, such
as word processing, emails etc.

\bigskip
\noindent\textbf{(2)} \textit{Describe the roles and basic functions of an operating system.}

This includes managing user files; handling interrupts; providing interface; managing peripherals
and drivers; managing memory; managing multitasking; providing applications a platform; providing
system security and managing user accounts.

\bigskip
\noindent\textbf{(3)} 
\textit{Understand how hardware, firmware and an operating system are required to
run application software.}

Applications are run on the operating system, which is run on the firmware (bootloader), which is
run on the firmware. The firmware runs on the hardware.

\bigskip
\noindent\textbf{(4)}
\textit{Understand the role and operation of interrupts.}

An interrupt is a signal sent to the computers processor to signal that something requires its
attention. It is generated when something has gone wrong, and can originate from software or 
hardware. Software interrupts include events such as division by zero and hardware interrupts may
be generated when two simultaneous inputs are registered, from two separate peripherals.

\subsection{Types of programming language, translators and integrated development environments
(IDEs)}
\textbf{1} \textit{Understand what is meant by a high level and low level language, understanding
the advantages and disadvantages of each}

High level languages uses words from human language. Keywords in such a language include 
\verb|if|, \verb|while|, etc.

Such a program is easier for humans to understand, read, write and amend. It is hence easier for
humans to debug the code. The code itself is portable and machine independent. A single high
level statement can represent multiple low level statemnts.

However, it is slower to work with in the sense that it must be converted to a low level language
before the code can be run. High level languages cannot usually directly manipulate hardware.

\medskip

Low level languages are of two types, machine code and assembly language. A computer executes
machine code consisting of 0s and 1s. This kind of code is non portable, the machine code that runs
on one machine may not run on another. Example of such code would be \verb|01001011|.

Assembly language is an in between stage, that uses mnemonics to represent code. For example
\verb|STO| can be used to store some memory at some address. All high level languages are 
translated to assembly before being converted to machine code. 

Low level languages need not be converted through multiple stages, machine code can be executed
directly and assembly can be translated directly to machine code with no in between stages, making
it a faster process. Such languages can directly manipulate hardware, by writing to specific
memory locations, making the program more efficient in terms of speed and memory usage.

Such a language is harder to read, write and amend, and hence harder to debug. Code written in this
language tends to be machine dependent. Several instructions are required per high level statement.

\bigskip

\noindent\textbf{2} \textit{Understand that assembly language is a form
of low-level language that uses mnemonics,
and that an assembler is needed to translate an
assembly language program into machine code.}

Statements in assembly look like:

\begin{center}
\begin{BVerbatim}
LDD count
ADD 1
STO count
\end{BVerbatim}
\end{center}
which use shortened words, mnemonics, to represent statements.

Assemblers are translator software which translate or convert assembly statements into machine code
that can then be executed by the computer.

\bigskip

\noindent\textbf{3} \textit{Describe the operation of a compiler and an
interpreter, including how high-level language is
translated by each and how errors are reported.}

High level languages are translated to machine code by means of translation software called
compilers and interpreters.

A compiler translates the whole code at once, producing an executable file, if there are no errors
in the code. If there are errors, all the errors in the file are reported together by the compiler.

An interpreter translates and executes the code line by line, and stops execution if a line with
an error is encountered.

\bigskip

\noindent\textbf{4} \textit{Explain the advantages and disadvantages of a
compiler and an interpreter.}

An interpreter is used during the development of a program. This allows users to see when an error
is encountered, amend it and continue running the program from the same point. 

Once the program has been developed, a compiler will be used, as this will produce a machine
independent executable file. The executable file can then be distributed, safely given that the
code is undecipherable from the binary of an executable file. An interpreter, in this situation
would not be appropriate as the source code and the interpreter software would both have to be
distributed, meaning users can also have access to the source code, being able to change it as they
like.

\noindent\textbf{5} \textit{Explain the role of an IDE in writing program code
and the common functions IDEs provide}

IDEs provide code editors, a run time environment, translators, error diagnostics, auto completion,
auto correction and pretty print.
