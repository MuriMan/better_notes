\section{Nuclear physics}
\subsection{The nuclear model of the atom}
\subsubsection{The atom}

\begin{subpoint}
Describe the structure of the atom in terms of a positively charged nucleus and negatively charged
electrons in orbit around the nucleus
\end{subpoint}

\begin{subpoint}
Describe how alpha-particle scattering experiments provide evidence for:
\begin{enumerate}[label=(\alph*)]
	\setlength\itemsep{0pt}
	\item a very small nucleus surrounded by mostly empty space
	\item a nucleus containing most of the mass of the atom
	\item a nucleus that is positively charged
\end{enumerate}
\end{subpoint}

Alpha ($\alpha$) particles are positively charged. They are emitted by certain sources.
When emitted at a gold leaf, it is seen that almost all of the alpha particles continue
in their straight path. 

However, some particles are \emph{deflected} and travel not in
that straight line. This happens when the particle comes close to a nucleus of the
gold atoms. The positive charges of the nucleus and alpha particle cause the alpha 
particle's trajectory to change due to repulsion.

These observations prove that, 
\begin{enumerate}[label=(\roman*)]
	\item \ul{\emph{Most of the atom is empty space}} since the majority of alpha
		particles travel in the trajectory they were sent in as.
	\item \emph{\ul{The nucleus contains the mass of the atom}} because empty space
		can't have weight.
	\item \emph{\ul{The nucleus is positively charged}} because it repulses the 
		positively charged alpha particles (like charges repulse).
\end{enumerate}

\subsubsection{The nucleus}
\begin{subpoint}
Describe the composition of the nucleus in terms of protons and neutrons
\end{subpoint}

The nucleus consists of two types of massive\footnote{In physics, \emph{massive} means anything
that has mass, not that anything has a lot of mass.}, subatomic particles. These are
\emph{\ul{protons}} -- particles that have a positive charge, of \SI{1.602e-19}{C},
and \emph{\ul{neutrons}} -- particles which have no charge, but mass equal to that of
a proton.

The masses of these particles, in actuality is \SI{1.6727e-24}{g}. A carbon-12 atom has
12 of these particles. Nuclear masses are measured relative to 1/12th that of a 
carbon-12 atom, so a neutron and proton have \emph{relative masses} of 1 each. Charges
are also measured relative to the charge of 1/6th of the charge of a carbon-12 nucleus.
\newpage

\begin{subpoint}
Describe how atoms form positive ions by losing electrons or negative ions by gaining electrons
\end{subpoint}

Protons, since they are massive and are present at the centre of an atom, they do not
always move. Charge in an atom is brought about by the movement of 
\emph{\ul{electrons}}. These are particles that have opposite the charge of a proton,
i.e., \SI{-1.602e-19}{C}, or a relative charge of -1. These particles have immensely
negligible mass and they orbit the nucleus, hence they can move around easily.

An \emph{excess of electrons} causes a negative charge, and an 
\emph{absence of electrons} causes a positive charge, since the resulting sums of
charges are negative and positive respectively.

\begin{subpoint}
Define the terms proton number (atomic number) $Z$ and nucleon number (mass number) $A$ and be able to
calculate the number of neutrons in a nucleus
\end{subpoint}

The proton number of a given nucleus is denoted $Z$, and is the number of protons in
that nucleus. The nucleon number of a nucleus is denoted $A$, it is also called mass
number and it is the total number of protons and neutrons in that nucleus. So, for
the number of neutrons in the nucleus:
$$ A - Z $$

\begin{subpoint}
Explain the term nuclide and use the nuclide notation $^A_Z X$
\end{subpoint}

A \emph{\ul{nuclide}} is simply a nucleus whose details are known. The nucleon and
proton numbers of a nuclide X, $A$ and $Z$ is denoted $^A_Z$X.

\begin{subpoint}
Explain what is meant by an isotope and state that an element may have more than one isotope
\end{subpoint}

An \emph{\ul{isotope}} of an element is an atom of that element with a certain number
of neutrons. Multiple of these atomic variations may exist per element.

\subsection{Radioactivity}
\subsubsection{Detection of radioactivity}

\begin{subpoint}
Describe the detection of alpha particles ($\alpha$-particles) using a cloud chamber or spark counter and the
detection of beta particles ($\beta$-particles) ($\beta$-particles will be taken to refer to $\beta^-$) and gamma radiation
($\gamma$-radiation) by using a Geiger-Müller tube and counter
\end{subpoint}

Spark counters, are weird. Geiger Muller tubes are things that detect radiation and
counters calculate and display the detections as counts/min.

\begin{subpoint}
Use count rate measured in counts/s or counts/minute
\end{subpoint}

\begin{subpoint}
Know what is meant by background radiation
\end{subpoint}

Radiation present from natural sources in the atmosphere is called \emph{\ul{background
radiation}}.

\begin{subpoint}
Know the sources that make a significant contribution to background radiation including:
\begin{enumerate}[label=(\alph*)]
	\setlength\itemsep{0em}
	\item radon gas (in the air)
	\item rocks and buildings
	\item food and drink
	\item cosmic rays
\end{enumerate}
\end{subpoint}

Food and drink are gotten from the soil, which can contain radioactive sources.

\begin{subpoint}
Use measurements of background radiation to determine a corrected count rate
\end{subpoint}

When measuring radiation from a source, the count rate will show both background
radiation and the radiation from the source itself. So, to get the radiation from 
only the source, first, the background radiation must be measured itself, then
subtracted from the reading gotten from the source.

\subsubsection{The three types of emission}
\begin{subpoint}
Describe the emission of radiation from a nucleus as spontaneous and random in direction
\end{subpoint}

Spontaneous, here, refers to the fact that it is unpredictable and cannot be influenced,
the time of emission of radiation from a nucleus.

\begin{subpoint}
Describe $\alpha$-particles as two protons and two neutrons (helium nuclei), 
$\beta$-particles as high speed electrons from the nucleus and $\gamma$-radiation as
high-frequency electromagnetic waves
\end{subpoint}

\begin{subpoint}
State, for $\alpha$-particles, $\beta$-particles and $\gamma$-radiation:
\begin{enumerate}[label=(\alph*)]
	\setlength\itemsep{0em}
	\item their relative ionising effects
	\item their relative penetrating powers
\end{enumerate}
\end{subpoint}

For $\alpha$, $\beta$ and $\gamma$, in that order:
\begin{itemize}
	\item Alpha is the most ionising, gamma is the least.
	\item Alpha is the least penetrating, gamma is the most.
\end{itemize}

\begin{subpoint}
Describe the deflection of $\alpha$-particles, $\beta$-particles and $\gamma$-radiation in electric fields and magnetic fields
\end{subpoint}

Gamma radiation has no charge, so is unaffected in both electric and magnetic fields.

Direction of movement of alpha particles is the same as the movement of conventional
current (movement of positive charge) and that of beta particles is opposite of 
conventional current (movement of electrons). Using this information, we can apply
left hand rule in magnetic fields to find the deflections (forces) acting on these
particles.

Since alpha particles are more massive, their relative deflection is less (greater
moving inertia) than beta particles.

\subsubsection{Radioactive decay}
\begin{subpoint}
Know that radioactive decay is a change in an unstable nucleus that can result in the
emission of $\alpha$-particles or $\beta$-particles and/or $\gamma$-radiation and know
that these changes are spontaneous and random
\end{subpoint}

The instability of nuclei arises when a nucleus is too energetic or is too heavy. To
shed this extra energy/mass, radiation is emitted in the form of alpha, beta or gamma.

\begin{subpoint}
Use decay equations, using nuclide notation, to show the emission of 
$\alpha$-particles, $\beta$-particles and $\gamma$-radiation
\end{subpoint}

\begin{subpoint}
Use decay equations, using nuclide notation, to show the emission of 
$\alpha$-particles $\beta$-particles and $\gamma$-radiation
\end{subpoint}

The nuclides for alpha and beta particles are as such: $^4_2\alpha$ and $^0_{-1}\beta$.
Gamma has no nuclide notation because it is not a nuclide.

Let's observe the emission of alpha and beta radiation from a theoretical nuclide, X, 
with atomic and nucleon numbers of 15 and 30 respectively.

$$ \ce{ ^{30}_{15}X -> ^{26}_{13}Y + ^4_2\alpha } $$
$$ \ce{ ^{30}_{15}X -> ^{30}_{16}Y + ^0_{-1}\beta } $$

\subsubsection{Fission and fusion}
\begin{subpoint}
Describe the process of fusion as the formation of a larger nucleus by combining two smaller nuclei with
the release of energy, and recognise fusion as the energy source for stars
\end{subpoint}

\begin{subpoint}
Describe the process of fission when a nucleus, such as uranium-235 (U-235), absorbs a neutron and
produces daughter nuclei and two or more neutrons with the release of energy
\end{subpoint}

\begin{subpoint}
Explain how the neutrons produced in fission create a chain reaction and that this is controlled in a nuclear
reactor, including the action of coolant, moderators and control rods
\end{subpoint}

In a nuclear reactor, the release of energy during fission is extracted in a controlled
manner. Uranium nuclei are arranged such that, when an initial neutron is thrown at
one nucleus, the neutrons resulting from that fission go and cause fission of more
nuclei, which repeats on and on.

The control of this \emph{\ul{chain reaction}}, comes down to the use of coolant,
moderators and control rods. Fission releases tremendous amount of heat, which is 
controlled by the coolant as need be. Moderators slow down the neutrons released during
fusion which can then be absorbed easier by control rods, hence controlling the
rate of nuclear fission.

\subsubsection{Half-life}

\begin{subpoint}
Define the half-life of a particular isotope as the time taken for half the nuclei of that isotope in any sample
to decay; recall and use this definition in calculations, which may involve information in tables or decay
curves
\end{subpoint}

For a given sample of a radioactive isotope, the time taken for the sample's mass to
halve is known. So, given a sample of mass $m$, with half life $h$, which has been
decaying for time $t$, the final mass $m_f$ will be:
$$ m_f = (m)\left(\frac{1}{2^{t/h}}\right) $$


\begin{subpoint}
Describe the dating of objects by the use of $^{14}$C
\end{subpoint}

Carbon-14 is a radioactive isotope of carbon. To find the age of ancient artifacts,
the amount of carbon-14 present is compared to the initial amount present, to see
how many half lives have passed and hence how much time has passed.

\begin{subpoint}
Explain how the type of radiation emitted and the half-life of the isotope determine
which isotope is used for applications including:
\begin{enumerate}[label=(\alph*)]
	\setlength\itemsep{0em}
	\item household fire (smoke) alarms
	\item irradiating food to kill bacteria
	\item sterilisation of equipment using gamma rays
	\item measuring and controlling thicknesses of materials with the choice of
		radiations used linked to penetration and absorption
	\item diagnosis and treatment of cancer using gamma rays
\end{enumerate}
\end{subpoint}

\emph{\ul{Household fire (smoke) alarms}}. Alpha particles travel only few centimetres
in air. When smoke is in the way, that distance reduces further. In household smoke
detectors, alpha particles are being continuously emitted and detected, and when
smoke is in the way, the rate of detection lowers, signalling the presence of smoke
and the alarm is set off.

\emph{\ul{Irradiating of food to kill bacteria}}. Food goes bad because of microbes
that decompose them. Intense exposure to gamma radiation tends to kill single celled 
organisms. Food, in this way is irradiated, and made sterile. Such food lasts longer
due to absence of microbes.

\emph{\ul{Sterilisation of equipment using gamma rays}}. In the same way as food
irradiation, medical equipment can be made free of microbes by exposing them to
intense gamma radiation.

\emph{\ul{Measuring and controlling thicknesses of materials with the choice of
radiations used linked to penetration and absorption}}. In the production of paper
or metal sheets, thickness of the sheets is made constant by use of penetration of
beta radiation. Beta radiation is sent across the sheet, and decrease and increase
in detection of the radiation signals the sheet being too thick and too thin,
respectively. Alpha and gamma radiation is unsuitable for this since they would be
absorbed completely or unaffected entirely, respectively.

\emph{\ul{Diagnosis and treatment of cancer using gamma rays}}. Cancer cells are often
killed by directing intense gamma rays at the growth, killing them as with microbes.

\subsubsection{Safety precautions}
\begin{subpoint}
State the effects of ionising nuclear radiations on living things, including cell death, mutations and cancer
\end{subpoint}

Radiation causes ionisation, which, when happens to the molecules of living tissue,
causes the cells to die, as if they had been burned. These are called \emph{radiation
burns}. The DNA in the cell may be affected, and they will hence change. This is
called \emph{mutation}. Mutations in cell DNA may result in the growth of a cancer.

\begin{subpoint}
	Explain how radioactive materials are moved, used and stored in a safe way, with
	reference to:
	\begin{enumerate}[label=(\alph*)]
		\setlength\itemsep{0em}
		\item reducing exposure time
		\item increasing distance between source and living tissue
		\item use of shielding to absorb radiation
	\end{enumerate}
\end{subpoint}
