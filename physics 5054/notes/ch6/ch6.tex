\section{Space physics}
\subsection{Earth and the Solar System}
\subsubsection{The Earth}

\begin{subpoint}
Know that:

\begin{enumerate}[label=(\alph*)]
\item the Earth is a planet that orbits the Sun once in approximately 365 days
\item the orbit of the Earth around the Sun is an ellipse which is approximately circular
\item the Earth rotates on its axis, which is tilted, once in approximately 24 hours
\item it takes approximately one month for the Moon to orbit the Earth
\item it takes approximately 500 s for light from the Sun to reach the Earth
\end{enumerate}
\end{subpoint}

\begin{subpoint}
	Define average orbital speed from the equation:
	$$ v = 2\pi r / T$$
	where $r$ is the average radius of the orbit and $T$ is the orbital period;
	recall and use this equation
\end{subpoint}

Orbital period refers to the time taken for the body to orbit.

\subsubsection{The Solar System}
\begin{subpoint}
Describe the Solar System as containing:
\begin{enumerate}[label=(\alph*)]
\setlength\itemsep{0em}
\item one star, the Sun
\item the eight named planets and know their order from the Sun
\item minor planets that orbit the Sun, including dwarf planets such as Pluto and asteroids in the asteroid
belt
\item moons, that orbit the planets
\item smaller Solar System bodies, including comets and natural satellites
\end{enumerate}
\end{subpoint}

In order of increasing distance from the Sun, the planets in the solar system are:
Mercury, Venus, Earth, Mars, Jupiter, Saturn, Uranus and Neptune.

\begin{subpoint}
Analyse and interpret planetary data about orbital distance, orbital period, density, surface temperature
and uniform gravitational field strength at the planet’s surface
\end{subpoint}\newpage

\begin{subpoint}
Know that the strength of the gravitational field:

\begin{enumerate}[label=(\alph*)]
\setlength\itemsep{0em}
\item at the surface of a planet depends on the mass of the planet
\item around a planet decreases as the distance from the planet increases
\end{enumerate}
\end{subpoint}

\begin{subpoint}
Know that the Sun contains most of the mass of the Solar System and that the strength of the gravitational
field at the surface of the Sun is greater than the strength of the gravitational field at the surface of the
planets
\end{subpoint}

\begin{subpoint}
Know that the force that keeps an object in orbit around the Sun is the gravitational attraction of the Sun
\end{subpoint}

\begin{subpoint}
Know that the strength of the Sun’s gravitational field decreases and that the orbital speeds of the planets
decrease as the distance from the Sun increases
\end{subpoint}

\subsubsection{The Sun as a star}
\begin{subpoint}
Know that the Sun is a star of medium size, consisting mostly of hydrogen and helium, and that it radiates
most of its energy in the infrared, visible and ultraviolet regions of the electromagnetic spectrum
\end{subpoint}
\begin{subpoint}
Know that stars are powered by nuclear reactions that release energy and that in stable stars the nuclear
reactions involve the fusion of hydrogen into helium
\end{subpoint}

\subsubsection{Stars}
\begin{subpoint}
State that:
\begin{enumerate}[label=(\alph*)]
	\setlength\itemsep{0em}
	\item galaxies are each made up of many billions of stars
	\item the Sun is a star in the galaxy known as the Milky Way
	\item other stars that make up the Milky Way are much further away from the Earth
		than the Sun is from the Earth
	\item astronomical distances can be measured in light-years, where on light-year is
		the distance travelled in a vacuum by light in one year
\end{enumerate}
\end{subpoint}

\begin{subpoint}
Describe the life cycle of a star:
\begin{enumerate}[label=(\alph*)]
	\setlength\itemsep{0em}
\item a star is formed from interstellar clouds of gas and dust that contain hydrogen
\item a protostar is an interstellar cloud collapsing and increasing in temperature as a result of its internal
gravitational attraction
\item a protostar becomes a stable star when the inward force of gravitational attraction is balanced by an
outward force due to the high temperature in the centre of the star
\item all stars eventually run out of hydrogen as fuel for the nuclear reaction
\item most stars expand to form red giants and more massive stars expand to form red supergiants when
most of the hydrogen in the centre of the star has been converted to helium
\item a red giant from a less massive star forms a planetary nebula with a white dwarf at its centre
\item a red supergiant explodes as a supernova, forming a nebula containing hydrogen and new heavier
elements, leaving behind a neutron star or a black hole at its centre
\item the nebula from a supernova may form new stars with orbiting planets
\end{enumerate}
\end{subpoint}

\subsubsection{The Universe}
\begin{subpoint}
Know that the Milky Way is one of many billions of galaxies making up the Universe and that the diameter
of the Milky Way is approximately 100 000 light-years
\end{subpoint}

\begin{subpoint}
Describe redshift as an increase in the observed wavelength of electromagnetic radiation emitted from
receding stars and galaxies
\end{subpoint}

\begin{subpoint}
Know that the light from distant galaxies shows redshift and that the further away the galaxy, the greater
the observed redshift and the faster the galaxy’s speed away from the Earth
\end{subpoint}

\begin{subpoint}
Describe, qualitatively, how redshift provides evidence for the Big Bang theory
\end{subpoint}

It is seen that the redshift of objects, when observed after a time gap, is increasing
at an increasing rate. This means that the objects are accelerating outward, working
as proof for the Big Bang.
