\section{Motion, forces and energy}
\subsection{Physical quantities and measurement techniques}

\begin{firstpoint}
Describe how to measure a variety of lengths with appropriate precision using tapes, rulers and 
micrometers (including reading the scale on an analogue micrometer)
\end{firstpoint}

Every ruler has a minimum length it can measure. Readings must be written rounded to that nearest
reading. For a ruler calibrated to the nearest 0.1 mm, all readings should be written to the 
nearest 0.1 mm.
To read micrometers, watch: \url{https://youtu.be/StBc56ZifMs}.

\smallskip
Understand that, given a set of equipment and something to measure, we must choose the equipment
with maximum readings closest to the thing to measure.

\begin{syllpoint}
Describe how to use a measuring cylinder to measure the volume of a liquid and to determine the volume 
of a solid by displacement
\end{syllpoint}

For colourless liquids, readings in a measuring cylinder must be taken from the lower meniscus,
and for coloured liquids the upper meniscus should be used, see Figures 1 and 2.


\begin{figure}
	\centering

	\begin{tikzpicture}
		\draw (0, 0) -- (0, 4);
		\draw (2, 0) -- (2, 4);
		\draw (0, 2.5) arc(180:360:1);
		\draw[dashed] (2, 1.5) -- (0, 1.5) node[left]{reading level};
	\end{tikzpicture}

	\caption{For colourless liquids.}
\end{figure}

\begin{figure}
	\centering

	\begin{tikzpicture}
		\draw (0, 0) -- (0, 4);
		\draw (2, 0) -- (2, 4);
		\draw (0, 2.5) arc(180:360:1);
		\draw[dashed] (0, 2.5) -- (2, 2.5) node[right]{reading level};
	\end{tikzpicture}

	\caption{For coloured liquids.}
\end{figure}

\begin{syllpoint}
Describe how to measure a variety of time intervals using clocks and digital timers
\end{syllpoint}

\begin{syllpoint}
Determine an average value for a small distance and for a short interval of time by measuring multiples 
(including the period of oscillation of a pendulum)
\end{syllpoint}

For experiments involving oscillation, oscillations are counted, and total time for those 
oscillations is taken. The total time is divided by the number of oscillations, giving an average
value for each oscillation, removing influence of inaccurate and anomalous experiment.

\begin{syllpoint}
Understand that a scalar quantity has magnitude (size) only and that a vector quantity has magnitude and 
direction
\end{syllpoint}

\begin{syllpoint}
Know that the following quantities are scalars: distance, speed, time, mass, energy and temperature
\end{syllpoint}

\begin{syllpoint}
Know that the following quantities are vectors: displacement, force, weight, velocity, acceleration, 
momentum, electric field strength and gravitational field strength
\end{syllpoint}

\begin{syllpoint}
Determine, by calculation or graphically, the resultant of two vectors at right angles 
\end{syllpoint}

Mathematically, the resultant, $\bm{c}$, of two vectors, $\bm{a}$ and $\bm{b}$ has is given by:
$$ \bm{c} = \sqrt{\bm{a}^2 + \bm{b}^2} $$

The angle with the horizontal, $\theta_h$ and that with the vertical $\theta_v$ can be found:
$$ \tan{\theta_v} = \bm{b}/\bm{a} $$
$$ \tan{\theta_h} = \bm{a}/\bm{b} $$
refer to Figure 3 and 4 for directionality.

\begin{figure}
	\centering
	\begin{tikzpicture}[decoration={
		markings,
		mark=at position 0.5 with {\arrow{>}}}
		]

		\draw[->] (0, 0) -- (0, 2) node[midway, left] {$\bm{a}$};
		\draw[->] (0, 0) -- (4, 0) node[midway, below] {$\bm{b}$};
		\draw[dashed] (0, 2) -- (4, 2);
		\draw[dashed] (4, 0) -- (4, 2);
		\draw[postaction=decorate] (0, 0) -- (4, 2) node[midway, above]{$\bm{c}$};
		\draw (1.1, 0) node[right, above]{$\theta_h$};
		\draw(0, 0.5) node[right]{$\theta_v$};
	\end{tikzpicture}
	\caption{Vectors pointing outward.}
\end{figure}

\begin{figure}
	\centering
	\begin{tikzpicture}[decoration={
		markings,
		mark=at position 0.5 with {\arrow{>}}}
		]

		\draw[postaction=decorate] (0, 2) -- (0, 0) node[midway, left] {$\bm{a}$};
		\draw[postaction=decorate] (4, 0) -- (0, 0) node[midway, below] {$\bm{b}$};
		\draw[dashed] (0, 2) -- (4, 2);
		\draw[dashed] (4, 0) -- (4, 2);
		\draw[postaction=decorate] (4, 2) -- (0, 0) node[midway, below]{$\bm{c}$};
		\draw (1.1, 0) node[right, above]{$\theta_h$};
		\draw(0, 0.5) node[right]{$\theta_v$};
	\end{tikzpicture}
	\caption{Vectors pointing inward.}
\end{figure}

\subsection{Motion}
\begin{firstpoint}
Define speed as distance travelled per unit time and define velocity as change in displacement per unit time
\end{firstpoint}

Displacement is the distance of an object with respect to a certain point, called the origin.
Essentially, displacement is the vector form of distance. Speed and velocity, $\bm{v}$ are the rates
of change of distance and displacement, $\bm{s}$ with respect to time, $t$.

Mathematically,
$$ \bm{v} = \bm{s}/t $$

\begin{syllpoint}
Recall and use the equation
\end{syllpoint}

$$ (\textrm{average speed}) = (\textrm{total distance})/(\textrm{time taken}) $$

\begin{syllpoint}
Define acceleration as change in velocity per unit time; recall and use the equation
\end{syllpoint}

$$ (\textrm{acceleration}) = (\textrm{change in velocity})/(\textrm{time taken}) $$
Symbolically,
$$ \bm{a} = \frac{\Delta \bm{v}}{\Delta t} = \frac{\bm{v} - \bm{u}}{t}$$
where $\bm{v}$ is final and $\bm{u}$ is initial velocity.

\begin{syllpoint}
State what is meant by, and describe examples of, uniform acceleration and non-uniform acceleration
\end{syllpoint}

When over a period of time, acceleration does not change, i.e., $\Delta \bm{a} = 0$, acceleration
is said to be \textit{uniform} or \textit{constant}. When $\Delta \bm{a} \ne 0$, acceleration
has changed, causing a \textit{non-uniform} or \textit{non-constant} acceleration.

\begin{syllpoint}
Know that a deceleration is a negative acceleration and use this in calculations
\end{syllpoint}

A negative acceleration causes a decrease in velocity, and is called deceleration. A deceleration
of $x$ is the same as an acceleration of $-x$, and the opposite also applies.

\begin{syllpoint}
Sketch, plot and interpret distance–time and speed–time graphs
\end{syllpoint}

The motion of objects can be investigated by their data, which is made more convenient by the use 
of graphical representations of the data. The distance covered by the object and the velocity or
speed of the object can be plotted against time.

\begin{syllpoint}
Determine from the shape of a distance–time graph when an object is:

\indent(a) at rest
\indent(b) moving with constant speed
\indent(c) accelerating
\indent(d) decelerating
\end{syllpoint}

\begin{figure}
	\centering
	\begin{tikzpicture}

	\end{tikzpicture}
	\caption{Distance-time or speed-time graph.}
\end{figure}
