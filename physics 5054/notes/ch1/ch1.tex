\section{Motion, forces and energy}
\subsection{Physical quantities and measurement techniques}

\begin{point}
Describe how to measure a variety of lengths with appropriate precision using tapes, rulers and 
micrometers (including reading the scale on an analogue micrometer)
\end{point}

Every ruler has a minimum length it can measure. Readings must be written rounded to that nearest
reading. For a ruler calibrated to the nearest 0.1 mm, all readings should be written to the 
nearest 0.1 mm.
To read micrometers, watch \href{https://youtu.be/StBc56ZifMs}{\color{blue}this video}.

\smallskip
Understand that, given a set of equipment and something to measure, we must choose the equipment
with maximum readings closest to the thing to measure.

\begin{point}
Describe how to use a measuring cylinder to measure the volume of a liquid and to determine the volume 
of a solid by displacement
\end{point}

For colourless liquids, readings in a measuring cylinder must be taken from the lower meniscus,
and for coloured liquids the upper meniscus should be used, see Figures 1 and 2.


\begin{figure}
	\centering

	\begin{tikzpicture}
		\draw (0, 0) -- (0, 4);
		\draw (2, 0) -- (2, 4);
		\draw (0, 2.5) arc(180:360:1);
		\draw[dashed] (2, 1.5) -- (0, 1.5) node[left]{reading level};
	\end{tikzpicture}

	\caption{For colourless liquids.}
\end{figure}

\begin{figure}
	\centering

	\begin{tikzpicture}
		\draw (0, 0) -- (0, 4);
		\draw (2, 0) -- (2, 4);
		\draw (0, 2.5) arc(180:360:1);
		\draw[dashed] (0, 2.5) -- (2, 2.5) node[right]{reading level};
	\end{tikzpicture}

	\caption{For coloured liquids.}
\end{figure}

\begin{point}
Describe how to measure a variety of time intervals using clocks and digital timers
\end{point}

\begin{point}
Determine an average value for a small distance and for a short interval of time by measuring multiples 
(including the period of oscillation of a pendulum)
\end{point}

For experiments involving oscillation, oscillations are counted, and total time for those 
oscillations is taken. The total time is divided by the number of oscillations, giving an average
value for each oscillation, removing influence of inaccurate and anomalous experiment.

\begin{point}
Understand that a scalar quantity has magnitude (size) only and that a vector quantity has magnitude and 
direction
\end{point}

\begin{point}
Know that the following quantities are scalars: distance, speed, time, mass, energy and temperature
\end{point}

\begin{point}
Know that the following quantities are vectors: displacement, force, weight, velocity, acceleration, 
momentum, electric field strength and gravitational field strength
\end{point}

\begin{point}
Determine, by calculation or graphically, the resultant of two vectors at right angles 
\end{point}

Mathematically, the resultant, $\bm{c}$, of two vectors, $\bm{a}$ and $\bm{b}$ has is given by:
$$ \bm{c} = \sqrt{\bm{a}^2 + \bm{b}^2} $$

The angle with the horizontal, $\theta_h$ and that with the vertical $\theta_v$ can be found:
$$ \tan{\theta_v} = \bm{b}/\bm{a} $$
$$ \tan{\theta_h} = \bm{a}/\bm{b} $$
refer to Figure 3 and 4 for directionality.

\begin{figure}
	\centering
	\begin{tikzpicture}[decoration={
		markings,
		mark=at position 0.5 with {\arrow{>}}}
		]

		\draw[->] (0, 0) -- (0, 2) node[midway, left] {$\bm{a}$};
		\draw[->] (0, 0) -- (4, 0) node[midway, below] {$\bm{b}$};
		\draw[dashed] (0, 2) -- (4, 2);
		\draw[dashed] (4, 0) -- (4, 2);
		\draw[postaction=decorate] (0, 0) -- (4, 2) node[midway, above]{$\bm{c}$};
		\draw (1.1, 0) node[right, above]{$\theta_h$};
		\draw(0, 0.5) node[right]{$\theta_v$};
	\end{tikzpicture}
	\caption{Vectors pointing outward.}
\end{figure}

\begin{figure}
	\centering
	\begin{tikzpicture}[decoration={
		markings,
		mark=at position 0.5 with {\arrow{>}}}
		]

		\draw[postaction=decorate] (0, 2) -- (0, 0) node[midway, left] {$\bm{a}$};
		\draw[postaction=decorate] (4, 0) -- (0, 0) node[midway, below] {$\bm{b}$};
		\draw[dashed] (0, 2) -- (4, 2);
		\draw[dashed] (4, 0) -- (4, 2);
		\draw[postaction=decorate] (4, 2) -- (0, 0) node[midway, below]{$\bm{c}$};
		\draw (1.1, 0) node[right, above]{$\theta_h$};
		\draw(0, 0.5) node[right]{$\theta_v$};
	\end{tikzpicture}
	\caption{Vectors pointing inward.}
\end{figure}

\subsection{Motion}
\begin{point}
Define speed as distance travelled per unit time and define velocity as change in displacement per unit time
\end{point}

\ul{\emph{Displacement}} 
is the distance of an object with respect to a certain point, called the origin.
Essentially, displacement is the vector form of distance. Speed and velocity, $\bm{v}$ are the rates
of change of distance and displacement, $\bm{s}$ with respect to time, $t$.

\begin{point}
Recall and use the equation
$$ \textrm{(speed)} = \textrm{(distance)}/\textrm{(time)} $$
\end{point}

Mathematically,
$$ \bm{v} = \bm{s}/t $$

\begin{point}
Recall and use the equation
$$ (\textrm{average speed}) = (\textrm{total distance})/(\textrm{time taken}) $$
\end{point}


\begin{point}
Define acceleration as change in velocity per unit time; recall and use the equation
$$ (\textrm{acceleration}) = (\textrm{change in velocity})/(\textrm{time taken}) $$
$$ \bm{a} = \frac{\Delta \bm{v}}{\Delta t}$$
\end{point}

Symbolically,
$$ \bm{a} = \frac{\Delta \bm{v}}{\Delta t} = \frac{\bm{v} - \bm{u}}{t}$$ where $\bm{v}$ is final and $\bm{u}$ is initial velocity.

\begin{point}
State what is meant by, and describe examples of, uniform acceleration and non-uniform acceleration
\end{point}

When over a period of time, acceleration does not change, i.e., $\Delta \bm{a} = 0$, acceleration
is said to be \ul{\textit{uniform}} or \underline{\textit{constant}}. 
When $\Delta \bm{a} \ne 0$, acceleration
has changed, causing a \textit{non-uniform} or \textit{non-constant} acceleration.

\begin{point}
Know that a deceleration is a negative acceleration and use this in calculations
\end{point}

A negative acceleration causes a decrease in velocity, and is called \ul{\emph{deceleration}}. A deceleration
of $x$ is the same as an acceleration of $-x$, and the opposite also applies.

\begin{point}
Sketch, plot and interpret distance–time and speed–time graphs
\end{point}

The motion of objects can be investigated by their data, which is made more convenient by the use 
of graphical representations of the data. The distance covered by the object and the velocity or
speed of the object can be plotted against time.

\begin{point}
Determine from the shape of a distance–time graph when an object is:

\begin{enumerate}[label=(\alph*)]
\setlength\itemsep{0em}
\item at rest
\item moving with constant speed
\item accelerating
\item decelerating
\end{enumerate}
\end{point}

\begin{figure}
	\centering
	\begin{tikzpicture}
		\coordinate (O) at (0, 0) node[below]{$O$};
		\coordinate (xMax) at (8, 0);
		\coordinate (yMax) at (0, 5);

		\draw[->] (O) -- (xMax) node[right]{$t$ (s)};
		\draw[->] (O) -- (yMax) node[above]{$s$ (m)};

		\draw (0, 0) -- (1, 0);
		\draw (1, 0) node[below]{$t_1$};

		\draw (1, 0) -- (2, 2);
		\draw[dashed] (2, 2) -- (2, 0) node[below]{$t_2$};

		\draw (2, 2) sin (4, 3);
		\draw[dashed] (4, 3) -- (4, 0) node[below]{$t_3$};

		\draw (7, 4) sin (4, 3);
		\draw[dashed] (7, 4) -- (7, 0) node[below]{$t_4$};
	\end{tikzpicture}

	\caption{Distance-time graph.}
\end{figure}

For a distance time graph, understand that the gradient of the curve gives speed of the object
being observed. This is further discussed in Section 1.2.11 onwards.

\smallskip
Observe Figure 5. 

For $ 0 \le t \le t_1$, the distance travelled by the object being observed is not changing. This
simply means it is not moving, is stationary and at rest.

For $ t_1 \le t \le t_2$, the distance travelled by the object is increasing. The nature of the
increase is to be observed. For this interval, the line is straight. This means the gradient is
constant, and for a distance-time curve the gradient is the object's speed. Hence, for 
$ t_1 \le t \le t_2 $, the object travels with constant, uniform speed.

For $t_2 \le t \le t_3$, the distance travelled increases still, but the nature of the increase
is different. If we were to image a tangent along the graph in the interval, we would see that
the steepness of the tangent decreases as the tangent travel rightward, i.e., as time increases.
Hence, here, the object travels with a decreasing speed.

For $t_3 \le t \le t_4$, we use the same approach as the previous interval. However, here we will
notice the tangent increasing in gradient. Therefore, here, the object travels with increasing
speed.

\begin{point}
Determine from the shape of a speed–time graph when an object is:
\begin{enumerate}[label=(\alph*)]
	\setlength\itemsep{0em}
	\item at rest
	\item moving with constant speed
	\item moving with constant acceleration
	\item moving with changing acceleration
\end{enumerate}
\end{point}

Understand that, for a speed-time graph, the gradient of the curve gives acceleration of the
observed object and the area under it gives the distance travelled by the object. This matter
is further discussed in Section 1.2.11 onwards.

\smallskip
Observe Figure 6, consider that an object's speed has been plotted against time.

\begin{figure}
	\centering
	\begin{tikzpicture} \coordinate (O) at (0, 0) node[below]{$O$};
		\coordinate (xMax) at (8, 0);
		\coordinate (yMax) at (0, 5);

		\draw[->] (O) -- (xMax) node[right]{$t$ (s)};
		\draw[->] (O) -- (yMax) node[above]{$v$ (m/s)};

		\draw (0, 0) -- (1, 0);
		\draw (1, 0) node[below]{$t_1$};

		\draw (1, 0) -- (2, 2);
		\draw[dashed] (2, 2) -- (2, 0) node[below]{$t_2$};

		\draw (2, 2) -- (3, 2);
		\draw[dashed] (3, 2) -- (3, 0) node[below]{$t_3$};

		\draw (3, 2) sin (5, 3);
		\draw[dashed] (5, 3) -- (5, 0) node[below]{$t_4$};

		\draw (7, 4) sin (5, 3);
		\draw[dashed] (7, 4) -- (7, 0) node[below]{$t_5$};
	\end{tikzpicture}

	\caption{Speed-time graph}
\end{figure}

For $0 \le t \le t_1$, notice that the graph is at the vertical coordinate of 0. That means speed
for this time is zero, and the object is not moving, i.e., is at rest.

For $t_1 \le t \le t_2$, the speed of the object has increased. The graph for this interval
is straight, meaning the speed has increased at a constant rate. The rate of change of velocity
is acceleration, and hence, the object has accelerated uniformly for this time interval.

For $t_2 \le t \le t_3$, the speed of the object does not change, but it has a non-zero value.
This means the object now travels at a uniform speed.

For $t_3 \le t \le t_4$, the speed of the object increases. The nature of the increase is to be
investigated. Observing the tangent as $t$ increases, we see its slope decreases, therefore,
here, the object travels with decreasing acceleration.

For $t_4 \le t \le t_5$, the speed of the object increases still. Observing the tangent to
the curve, we see its slope increases, hence, here the object travels with increasing acceleration.

\begin{point}
State that the acceleration of free fall g for an object near to the surface of the Earth is approximately
constant and is approximately 9.8 m/s$^2$
\end{point}

For an object near the surface of the Earth, it is attracted by the Earth. This force of attraction
is said to be \ul{\emph{gravity}}. An object affected by Earth's gravity, which is near Earth's
surface accelerates toward the
centre of the Earth at a rate of 9.8 m/s$^2$. This quantity, is called the \ul{\emph{acceleration
of free fall}} and is denoted mathematically as $g$.

\begin{point}
Calculate speed from the gradient of a distance–time graph
\end{point}

The gradient of a graph results in the division of the variable on the vertical axis divided by
the variable on the horizontal axis. The quantities' units too, are divided. Knowing this, we
see the gradient of a distance-time graph gives, (m)/(s) = m/s, the unit for speed.

The gradient at a point on a curve can be found by drawing a tangent to the point, taking two
convenient coordinates through which the tangent passes, let's say $(x_1, y_1)$ and $(x_2, y_2)$.
Hence the gradient would be:

$$ \frac{y_2 - y_1}{x_2 - x_1} = \frac{y_1 - y_2}{x_1 - x_1} $$

For a straight line, the line itself is its tangent, so taking two convenient points from it,
or extending it to get more convenient points, and applying them into the above formula gives
the gradient.

\begin{point}
Calculate the area under a speed–time graph to determine the distance travelled for motion with constant
speed or constant acceleration
\end{point}

The distance under a graph multiplies the horizontal and vertical variables, and hence their units.
In case of a speed-time graph, (m/s)(s) = m, the unit of distance. Therefore, area under a
speed-time graph gives distance travelled.

Geometry may be used to find the gradient under a graph.

\begin{point}
Calculate acceleration from the gradient of a speed–time graph
\end{point}

Using principles discussed in Section 1.2.11, the gradient of a speed time graph can be found,
which gives acceleration since (m/s)/(s) = m/s$^2$.

\subsection{Mass and weight}
