\section{The Periodic Table}
\subsection{Arrangement of elements}

\begin{point}
Describe the Periodic Table as an arrangement of elements in periods and groups and in order of increasing 
proton number/atomic number
\end{point}

\begin{figure}
	\caption{The Periodic Table of elements}
\end{figure}

The Periodic Table arranges all discovered elements by order of the number of their protons and
the number of electron shells they have. In the table, the \ul{\emph{columns are called groups
and rows are called periods}}.

\begin{point}
Describe the change from metallic to non-metallic character across a period
\end{point}

Across a period, from left to right, metallic character of elements decreases.

\begin{point}
Describe the relationship between group number and the charge of the ions formed from elements in that 
group
\end{point}

The Groups in the periodic table are labelled with the Roman numbers 1 through 8 (I through VIII).
The Group numbers are identical to the number of valence electrons the members of the Group
have, that is, members of Group $n$ have $n$ valence electrons.

\smallskip

For Group numbers less than four, it is easier for elements to lose their valence electrons
than to gain more than five electrons. Hence, these elements form positive ions with the size
of their charges identical to their group number.

For Group numbers above four, it is easier for elements to gain less than four electrons than
lose the electrons they have in their valence shell. As a result, elements with group number
$n$, where $n > 4$, the charges on their ions is $(n-8)$.

In the case of Group IV elements, it is equally easy to lose and gain the four valence electrons,
so elements in this group can form ions with $-4$ and $+4$ charge.

\begin{point}
Explain similarities in the chemical properties of elements in the same group of the Periodic Table in terms 
of their electronic configuration
\end{point}

Because all elements in the same Group have the same valence electrons, they tend to react
similarly and hence have similar or identical chemical properties.

\begin{point}
Explain how the position of an element in the Periodic Table can be used to predict its properties
\end{point}

For an element in the Periodic Table, its periodic position tells us about its metallic property
and the group it is in tells us about its chemical and often, physical properties.

\begin{point}
Identify trends in groups, given information about the elements
\end{point}

\subsection{Group I properties}
