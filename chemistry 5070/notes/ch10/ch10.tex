\section{Chemistry of the environment}
\subsection{Water}

\begin{point}
Describe chemical tests for the presence of water using anhydrous cobalt(II) chloride and anhydrous 
copper(II) sulfate
\end{point}

Anhydrous copper(II) sulfate, \chemfig{CuSO_4}, is white in colour. Upon addition of water, it
turns blue, as hydrated copper(II) sulfate, \chemfig{CuSO_4\cdot5H_2O} forms. In other words,
if anhydrous copper sulfate turns blue on addition of liquid being tested, the liquid contains
water.

Anhydrous cobalt(II) chloride, \chemfig{CoCl_2}, is blue. Upon addition of water it turns pink,
as hydrated cobalt(II) chloride, \chemfig{CoCl_2\cdot6H_2O}. In other words, if anhydrous cobalt
chloride turns pink on addition of liquid being tested, the liquid contains water.

\begin{point}
Describe how to test for the purity of water using melting point and boiling point
\end{point}

All substances have fixed melting and boiling points, which can be used to check for impurity. 
Impurities lower the melting point and raise the boiling point of a substance.

The melting point of water is 0 \celsius and its boiling point is 100 \celsius. Changes in these
values for a sample show impurity.

\begin{point}
Explain that distilled water is used in practical chemistry rather than tap water because it contains fewer 
chemical impurities
\end{point}

Chemical impurities in tap water may affect the outcomes of experiments, hence, in practical 
chemistry, distilled water is used.

\begin{point}
State that water from natural sources may contain substances, including:
\begin{enumerate}[label=(\alph*)]
	\setlength\itemsep{0em}
	\item dissolved oxygen
	\item metal compounds
	\item plastics
	\item sewage
	\item harmful microbes
	\item nitrates from fertilisers
	\item phosphates from fertilisers and detergents
\end{enumerate}
\end{point}

\begin{point}
State that some of these substances are beneficial, including:

\begin{enumerate}[label=(\alph*)]
	\setlength\itemsep{0em}
	
	\item dissolved oxygen for aquatic life
	\item some metal compounds provide essential minerals for life
\end{enumerate}
\end{point}

\begin{point}
State that some of these substances are potentially harmful, including:
\begin{enumerate}[label=(\alph*)]
	\setlength\itemsep{0em}
	\item some metal compounds are toxic
	\item some plastics harm aquatic life
	\item sewage contains harmful microbes which cause disease
	\item nitrates and phosphates lead to deoxygenation of water and damage to aquatic life
\end{enumerate}
\end{point}

\begin{point}
Describe the treatment of the domestic water supply in terms of:
\begin{enumerate}[label=(\alph*)]
	\setlength\itemsep{0em}
	\item sedimentation and filtration to remove solids
	\item use of carbon to remove tastes and odours
	\item chlorination to kill microbes
\end{enumerate}
\end{point}

\subsection{Fertilisers}

\begin{point}
State that ammonium salts and nitrates are used as fertilisers
\end{point}
\begin{point}
Describe the use of NPK fertilisers to provide the elements nitrogen, phosphorus and potassium for 
improved plant growth
\end{point}

\subsection{Air quality and climate}
\begin{point}
	State the composition of clean, dry air as approximately 78\% nitrogen, \chemfig{N_2}, 21\% oxygen, \chemfig{O_2}, and the
	remainder as a mixture of noble gases and carbon dioxide, \chemfig{CO_2}
\end{point}

\begin{point}
State the source of each of these air pollutants:
\begin{enumerate}[label=(\alph*)]
	\setlength\itemsep{0em}
	\item carbon dioxide from the complete combustion of carbon-containing fuels
	\item carbon monoxide and particulates from the incomplete combustion of carbon-containing fuels
	\item methane from the decomposition of vegetation and waste gases from digestion in animals
	\item oxides of nitrogen from car engines
	\item sulfur dioxide from the combustion of fossil fuels which contain sulfur compounds
\end{enumerate}
\end{point}

\begin{point}
State the adverse effects of these air pollutants:
\begin{enumerate}[label=(\alph*)]
	\setlength\itemsep{0em}
	\item carbon dioxide: higher levels of carbon dioxide leading to increased global warming, which leads to 
		climate change
	\item carbon monoxide: toxic gas
	\item particulates: increased risk of respiratory problems and cancer
	\item methane: higher levels of methane leading to increased global warming, which leads to
		climate change
	\item oxides of nitrogen: acid rain, photochemical smog and respiratory problems
	\item sulfur dioxide: acid rain
\end{enumerate}
\end{point}

\begin{point}
Describe how the greenhouse gases carbon dioxide and methane cause global warming, limited to:
\begin{enumerate}[label=(\alph*)]
	\setlength\itemsep{0em}
	\item the absorption, reflection and emission of thermal energy 
	\item reducing thermal energy loss to space
\end{enumerate}
\end{point}

Methane and carbon dioxide are \underline{\emph{greenhouse gases}}. They are called such because 
they enhance the \underline{\emph{greenhouse effect}}.

[ The greenhouse effect is where light sent by the Sun is absorbed by the Earth's atmosphere. This
happens because when the light is reflected off of the Earth's surface, its wavelength increases,
turning it into another form of light, \emph{infrared light}, which is absorbed by certain gases. ]

Greenhouse gases absorb the reflected rays of the Sun's light, reducing the amount of thermal
energy emmited and reflected out to space. To an extent, this is helpful as this is responsible
for keeping the almost constant temperature of the atmosphere. However, due to the increased 
amount of greenhouse gases in the atmosphere now, this is causing an \emph{enhanced} greenhouse
effect, which increase global temperatures further, causing \underline{\emph{global warming}}.

\begin{point}
State and explain strategies to reduce the effects of these environmental issues, limited to:
\begin{enumerate}[label=(\alph*)]
	\item climate change: planting trees, reduction in livestock farming, decreasing use of fossil fuels, increasing 
use of hydrogen and renewable energy, e.g. wind, solar
	\item acid rain: use of catalytic converters in vehicles, reducing emissions of sulfur dioxide by using low-
sulfur fuels and flue gas desulfurisation with calcium oxide
\end{enumerate}
\end{point}

\begin{point}
Explain how oxides of nitrogen form in car engines and describe their removal by catalytic converters, e.g. 
\schemestart
\chemfig{2CO + 2NO}\arrow{->}\chemfig{2CO_2 + N_2}
\schemestop
\end{point}

Carbon monoxide and nitrogen oxide are two harmful gases produced and released by car engines. To
reduce the extent of their damage, \underline{\emph{catalytic converters}} are used. In these
catalytic converters, the reaction:

\begin{center}
\schemestart
\chemfig{2CO + 2NO}\arrow{->}\chemfig{2CO_2 + N_2}
\schemestop
\end{center}
takes place, converting the two dangerous gases to carbon dioxide and diatomic nitrogen.

\begin{point}
Describe photosynthesis as the reaction between carbon dioxide and water to produce glucose and oxygen 
in the presence of chlorophyll and using energy from light
\end{point}

\underline{\emph{Photosynthesis}} 
is a chemical reaction performed by plants to produce glucose, which forms water as
a byproduct. Glucose is required for respiration, another chemical reaction, that releases energy.
It uses energy from light, trapped by \underline{\emph{chlorophyll}}, a pigment present in plants.

\begin{point}
State the word equation and symbol equation for photosynthesis
\end{point}

The word and symbol equations for photosynthesis follow, in that order:

\begin{center}
\schemestart
carbon dioxide + water
\arrow{->}
glucose + oxygen
\schemestop

\schemestart
\chemfig{CO_2 + H_2O}
\arrow{->}
\chemfig{C_6H_{12}O_6 + O_2}
\schemestop
\end{center}
