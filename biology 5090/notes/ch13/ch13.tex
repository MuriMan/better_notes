\section{Excretion}
\subsection{Excretion}

\begin{point}
Describe excretion as the removal of toxic materials and the waste products of metabolism from 
organisms
\end{point}

\begin{point}
State that carbon dioxide is a waste product of respiration, which is excreted through the lungs
\end{point}


\begin{point}
State that urea is a toxic waste product produced in the liver from the breakdown of excess amino acids
\end{point}

\subsection{Urinary system}

\begin{point}
Identify, on diagrams, the kidneys, ureters, bladder and urethra and state the function of each (the 
function of the kidney should be described simply as removing urea and excess salts and water from the 
blood as urine)
\end{point}

\ul{\emph{Kidneys}} remove urea and excess salts from the blood, forming a liquid called 
\ul{\emph{urine}}. This urine is \emph{carried} to the bladder by 
\ul{\emph{ureters}}. The bladder is where urine is temporarily stored before being 
expelled from the body through the \ul{\emph{urethra}}.

\begin{point}
Explain the need for excretion, limited to toxicity of urea
\end{point}

Urea is toxic, and hence must be excreted.

\begin{point}
Outline the function of a nephron and its associated blood vessels, limited to:
\begin{enumerate}[label=(\alph*)]
	\setlength\itemsep{0em}
	\item the role of the glomerulus in the filtration from the blood of water, glucose, urea and ions
	\item the role of the nephron in the reabsorption of all of the glucose, some of the ions and most of the 
		water back into the blood
	\item the formation of urine containing urea, excess water and excess ions 
		water back into the blood
\end{enumerate}
\end{point}

Kidneys are made up of many, much smaller tubules called \ul{\emph{nephrons}}. The
nephron consists of \ul{\emph{Bowman's capsule}}, \ul{\emph{tubules}}, the
\ul{\emph{loop of Henle}} and a \ul{\emph{collecting duct}}. 

\smallskip
In Bowman's capsule, there is a capillary called the \ul{\emph{glomerulus}} which is
a branch of the \ul{\emph{renal artery}}. This
capillary then branches out, and surrounds the rest of the nephron before joining together to
form part of the \ul{\emph{renal vein}}.

As blood flows into the glomerulus, small molecules such as water, glucose, ions and urea are 
filtered out of the blood and into the nephron through the capsule. As the filtered liquid passes
through the nephron, the useful substances that have been filtered out, such as glucose, water and
some ions are reabsorbed into the blood capillaries that surround the nephron. The resulting
liquid is urine, which then collects in a collecting duct, which eventually joins into the ureter.
Urine hence consists of excess water, excess ions and urea.

\begin{point}
Describe the role of the liver in the assimilation of amino acids by converting them to proteins
\end{point}

Amino acids, absorbed in the small intestine, are sent to be assimilated to the liver via the
hepatic portal vein (see Chapter 8). The liver assmilates amino acids by joining them together to
form new proteins.

\begin{point}
Describe deamination in the liver as the removal of the nitrogen-containing part of amino acids, resulting 
in the formation of urea
\end{point}
The removal of the nitrogen containing part of amino acids, is called 
\ul{\emph{deamination}}. This nitrogen containing part eventually forms urea.
\footnote{\emph{Excess amino acids cannot be stored. The nitrogen-containing part of amino acids is removed to
form a substance similar to sugars, which is converted to glycogen and stored.}}
