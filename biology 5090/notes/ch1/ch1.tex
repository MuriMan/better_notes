\section{Cells}
\subsection{Cell structure and function}

\begin{point}
Examine under the microscope, animal cells and plant cells from any suitable locally available material,
using an appropriate temporary staining technique, such as methylene blue or iodine solution
\end{point}

\begin{point}
Draw diagrams to represent observations of the animal and plant cells examined above
\end{point}

Make sure that the diagrams have a single smooth outline.

\begin{point}
Identify on diagrams, photomicrographs or electron micrographs, the ribosomes, mitochondria, nucleus,
cytoplasm and cell membrane in an animal cell
\end{point}

Ribosomes are very tiny structures present in the cytoplasm of an animal cell, which are
responsible for protein synthesis in an animal. Mitochondria are slightly larger, ovular 
structures, which are the site of aerobic respiration. Nuclei are significantly larger, circular
structures which contain the genetic material of the cell. The cytoplasm is a jelly like
substance, consisting mostly of water, in which the stated structures are suspended. The
cell membrane holds the cell cytoplasm, which holds the rest of the structures.

\begin{point}
Identify on diagrams, photomicrographs or electron micrographs, the ribosomes, mitochondria,
chloroplasts, nucleus, sap vacuole, cytoplasm, cell membrane and cellulose cell wall in a plant cell
\end{point}

Ribosomes, mitochondria, nuclei, cytoplasm and cell membrane are all identically present in plant
cells as they are in animal cells. Chloroplasts are medium sized green structures containing
the pigment chlorophyll which is responsible for absorption of energy from sunlight. The sap
vacuole is a large structure filled with cell sap. The cell wall is a protective layer of
carbohydrate (cellulose) that envelopes the plant cell.

\begin{point}
Describe the structure of a bacterial cell, limited to: ribosomes, circular deoxyribonucleic acid (DNA) and
plasmids, cytoplasm, cell membrane and cell wall
\end{point}

Ribosomes, cytoplasm, cell membrane and cell wall are identical in bacterial cells, only the
cell wall is not made of cellulose. In place of the genetic material containing nucleus, in
bacterial cells, there is circular DNA suspended freely in the cellular cytoplasm. In addition
to these, there are smaller circular lengths of DNA which are called plasmids.

\begin{point}
Describe the functions of the above structures in animal, plant and bacterial cells
\end{point}

\begin{itemize}
	\item \ul{\emph{Ribosomes}}: Tiny structures present in all cells. These are where proteins
		are made. ``Instructions'' on DNA molecules are read to join together amino acids, the
		monomer unit of a protein (see Chapter 4), to form proteins.
	\item \ul{\emph{Mitochondria}}: Structures that are larger than ribosomes, which are the
		site of aerobic respiration in plants, animals and bacteria. Respiration is a chemical
		reaction that releases energy, which can then be used for many other functions.
	\item \ul{\emph{Cell membrane}}: Cell membranes are a very thin layer of protein and fat,
		which are present around all cells. This membrane controls movement of substances in
		and out of the cells, and it is hence said to be \ul{\emph{partially permeable}}.
		A partially permeable membrane allows movement of some substances whilst inhibiting 
		movement of others.
	\item \ul{\emph{Cytoplasm}}: A clear jelly, consisting of mostly water. Many substances
		are dissolved in this jelly, and it is the site of most \emph{metabolic reactions}
		which are the reactions of life.
	\item \ul{\emph{Nucleus}}: Nuclei are relatively large structures that are absent in
		bacterial cells.
\end{itemize}
