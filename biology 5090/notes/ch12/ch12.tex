\section{Disease and immunity}
\subsection{Disease}

\begin{point}
Describe a pathogen as a disease-causing organism
\end{point}

Examples of such organisms include bacteria and viruses.

\begin{point}
Describe a transmissible disease as a disease in which the pathogen can be passed from one host to 
another
\end{point}

Examples include Covid-19, influenza and the common cold.

\begin{point}
Understand that a pathogen may be transmitted: 
\begin{enumerate}[label=(\alph*)]
\setlength\itemsep{0em}
\item through direct contact, including through blood or other body fluids
\item indirectly, including from contaminated surfaces or food, from animals, or from the air
\end{enumerate}
\end{point}

Direct contact comes down to exchange of bodily fluids. Indirect contact comes down to 
transmission of pathogens with an intermediary stage, with the pathogen moving through a medium, 
such as contaminated surfaces, animals, air etc.

\begin{point}
Describe the human body’s barriers to the entry of pathogens, limited to: skin, hairs in the nose, mucus, 
stomach acid
\end{point}

\begin{itemize}
\item\underline{Skin.}
Skin prevents pathogens' entrance. Blood clots broken skin to stop pathogens from getting into the
body and causing disease.

\item\underline{Hairs in the nose.} Particles in the air are filtered out by the nose hairs. These 
particles may contain pathogens.

\item\underline{Mucus.} Mucus in the airway traps pathogens. They are swept up to the back of the throat
and swallowed or spat out. This prevents the pathogens from entering the lungs. 

\item\underline{Stomach acid.} The stomach contains hydrochloric acid, which kills the pathogens that
are swallowed with food and the pathogens in swallowed mucus.
\end{itemize}


\begin{point}
Understand the role of the mosquito as a vector of disease
\end{point}

Often, diseases can be carried by mosquitoes. An example of such a disease is 
\underline{\emph{malaria}}. Mosquitoes bite infected individuals, drink their blood, in which lies
a pathogen. This pathogen, in some way, remains in the mosquito and is transmitted to the next
person who is bitten by that mosquito. In other words, mosquitoes carry and transmit the disease.
Such organisms are called \underline{\emph{vectors}}.

\begin{point}
Describe the malarial pathogen as an example of a parasite and explain how it is transmitted
\end{point}

A \underline{\emph{parasite}} is an organism that is incapable of independent existence and 
requires a \emph{host} to perform the functions of an organism (see Chapter 2). The malaria
pathogen is an example of such an organism. 

It is transmitted when an infected individual is bitten by a mosquito and that mosquito bites
another individual. What happens is, the parasite enters the mosquitoes body and survives
reproduces until it bites another, transferring the some of the 
pathogens from itself to the person.\newpage

\begin{point}
Describe the control of the mosquito that transmits malaria with reference to its life cycle
\end{point}

Since the medium of spread of malaria are the mosquitoes, controlling these mosquitoes will
result in the control of the disease. This may be done by many methods.

\begin{itemize}[]
\item\underline{Spraying insecticides.} Spraying insecticides around human settlements will kill
the mosquitoes that are already alive and which are potential carriers of the disease.

\item\underline{Spreading oil over water surfaces.} Mosquitoes lay their eggs in stagnant water
surfaces. Oil, spread over these surfaces, will float on top of the water, preventing the 
mosquitoes from laying their eggs.

\item\underline{Draining bodies of water.} Removing stagnant bodies of water leaves mosquitoes without
a place to lay their eggs.

\item\underline{Predatory fish.} Fish that feed on mosquito larvae can be used to populate bodies of
water. These will eat the larvae and reduce mosquitoes growing to maturity.
\end{itemize}

Sleeping under mosquito nets, taking malaria preventing drugs, vaccines that provide partial
immunity can all be used to \emph{prevent} malaria. Furthermore, male mosquitoes made infertile
by treatment with radiation may be released into the wild to produce infertile eggs.

\begin{point}
Explain that human immunodeficiency virus (HIV) is a viral pathogen
\end{point}

\begin{point}
Describe how HIV is transmitted
\end{point}

HIV spreads through direct contact. This includes unprotected sexual intercourse, blood 
transfusion, placental exchange and via breast milk.

\begin{point}
Understand that HIV infection may lead to Acquired Immune Deficiency Syndrome (AIDS)
\end{point}

HIV targets and destroys lymphocytes (see Section 12.3). This makes the body vulnerable in that
it will not be able to fight off infections. This state, where the body is susceptible to 
diseases, is called AIDS.

\begin{point}
Describe the methods by which HIV may be controlled
\end{point}

\begin{itemize}
\item\underline{Screening blood.} Before transfusion, blood to be transfused can be screened for the
HIV pathogen.

\item\underline{Reusing syringes.} Reusing syringes must be avoided as traces of blood and potentially
pathogens may remain on the needle.

\item\underline{Condoms or femidoms.} ``Protection'' should be used during intercouse.

\item\underline{Limited partners.} Multiple sexual partners should be avoided.

\item\underline{Awareness.} Awareness about the matter should be spread.
\end{itemize}

\begin{point}
Describe cholera as a disease caused by a bacterium, which is transmitted in contaminated water
\end{point}

Cholera is a disease that is transmitted indirectly via water.

\begin{point}
Explain the importance of a clean water supply, hygienic food preparation, good personal hygiene, waste 
disposal and sewage treatment in controlling the spread of cholera (details of the stages of sewage 
treatment are not required)
\end{point}

\begin{itemize}
\item\underline{A clean water supply.} Water is used for many daily purposes including drinking
	and cleaning. Water from an unclean source will always contain microorganisms, some of which
	may be pathogens. The cholera bacterium is such a pathogen. 

	A filtered and chlorinated water supply hence controls the spread of the disease.

\item\underline{Hygienic food preparation.} Hands must always be washed before touching food.
	Coughing and sneezing over food should be avoided. Hair should be carefully kept out of food.
	Animals should always be kept away from food. This is done to prevent bacteria from these places
	getting onto the food. Food should not be kept at room temperature for 
	too long as that would cause bacteria to grow. Raw meat contains some bacteria, and hence should
	be kept away from cooked meat and other foods.

\item\underline{Good personal hygiene.} The human skin produces oil, which, when not washed,
	can build up and particles such as dirt and microorganisms can accumulate in it. Such
	things provide bacterial breeding grounds. Prevention of this can be done by regular usage
	of soap and shampoo.

\item\underline{Waste disposal.} Waste may accumulate near settlements. Chemicals may seep out of
	the rubbish, polluting ground and water bodies. Food waste can provide a breeding ground
	for cholera. Landfill sites can be dug out to dispose of waste, which, when filled up, can
	be covered with soil and planted with trees.

\item\underline{Sewage treatment.} Raw sewage contains many bacteria, including cholera. People
	that contact these substances may get ill. So, this liquid must be treated before it is
	allowed to run into rivers.
\end{itemize}

\begin{point}
Explain that the cholera bacterium produces a toxin that causes secretion of chloride ions into the small 
intestine, causing osmotic movement of water into the gut, resulting in diarrhoea, dehydration and loss of 
ions from the blood
\end{point}

The toxin that the cholera bacterium secretes causes the epithelium (see Chapter 13) of the small
to secrete chloride ions into the lumen of the small intestine. This reduces the water potential
of the inside of the small intestine, causing water from the surrounding capillaries to flow out
into the intestinal lumen. This results in the patient suffering dehydration (lack of water), lack
of ions. The faeces becomes watery due to this excess movement of water. This is called
\emph{diarrhoea}.

\begin{point}
Describe the effects of excessive consumption of alcohol: reduced self-control, depressant, effect on 
reaction times, damage to liver and social implications
\end{point}

Alcohol is a \emph{depressant}, meaning it reduces the body's awareness by reducing the amount of
neurotransmitters that reach and stimulate nerves (see Chapter 14), this also increases reaction
times of the drunken individual. Society also looks down upon excessive alcohol consumers.

\begin{point}
Describe the effects of tobacco smoke and its major toxic components (nicotine, tar and carbon 
monoxide): strong association with bronchitis, emphysema, lung cancer, heart disease, and the 
association between smoking during pregnancy and reduced birth weight of the baby
\end{point}

\begin{itemize}
\item\underline{Nicotine.} Nicotine narrows blood vessels, increasing blood pressure. These
	narrowed vessels will be more easily clogged by fat, when this occurs in coronary arteries,
	it will lead to coronary heart disease (see Chapter 11). As a result, blood vessels will
	respire anaerobically, leading to a buildup of lactic acid in the heart muscles. This causes
	a low pH environment in the muscles, denaturing enzymes and eventually killing the muscle,
	which can lead to cardiac arrest. 

	Nicotine also narrows the blood vessels of the umbilical cord, reducing oxygen and nutrients
	reaching the baby. As a result, children of mothers who are smokers have lower birth weights.

\item\underline{Tar.} Tar is a \emph{carcinogen}, which is a substance that leads to an increased
	chance of development of cancerous cells. Tar does so for the lung cells.

	Tar destroys the tracheal cillia (see Chapter 9), which causes mucus to acumulate in the
	airways. This mucus often contains trapped pathogens. This may result in frequent infections.
	A smoker's cough is an attempt to remove this build up of mucus, but it only scars the
	epithelium of the airways. This scarring make breathing difficult.

	Emphysema is a result of frequent infection, meaning phagocytes are always near the lungs
	(see Section 12.3). These phagocytes release enzymes that break down the alveoli walls, 
	so that they can reach their target pathogens. This results in a reduced surface area for
	gas exchange. Some of these enzymes also reduce the elasticity of the alveoli, reducing their
	capacity to stretch during breathing, causing them to burst. As the condition progresses, the
	individual struggles to breathe.

\item\underline{Carbon monoxide.} Carbon monoxide binds to haemoglobin, reducing the blood's 
	capacity to carry oxygen. Breathing frequency and depth need to increase in order to get the
	same amount of oxygen into the blood. It also puts more strain on the circulatory system,
	increasing risk of heart diseases and strokes.
\end{itemize}

\subsection{Antibiotics}
\begin{point}
Describe a drug as any substance taken into the body that modifies or affects chemical reactions in the 
body
\end{point}

\begin{point}
Describe the use of antibiotics for the treatment of bacterial infection
\end{point}

\begin{point}
State that antibiotics kill bacteria but do not affect viruses
\end{point}

\begin{point}
Explain how development of antibiotic-resistant bacteria, including MRSA, can be minimised by using 
antibiotics only when essential
\end{point}

Antibiotic resistant bacteria develop as a result of mutations in bacterial populations, which
cause them to be unaffected by certain antibiotics. Over the past few decades, antibiotics have
been grossly overprescribed, leading to many antibiotic resistant bacterial strains developing.
These resistant strains are often called superbugs, an example of which is the MRSA superbug.

To combat the growing number of antibiotic resistant bacteria, antibiotics must only be prescribed
when necessary and prescribed courses must be completed by patients.

\subsection{Immunity}

\begin{point}
Describe active immunity as defence against a pathogen by antibody production in the body
\end{point}

\underline{\emph{Lymphocytes}} are a type of white blood cell that are responsible for the 
production of antibodies. \underline{\emph{Phagocytes}} are the other type of white blood cell
which can destroy and engulf pathogens by means of a process called 
\underline{\emph{phagocytosis}}.

\begin{point}
State that each pathogen has its own antigens, which have specific shapes
\end{point}

\begin{point}
Describe antibodies as proteins that bind to antigens leading to direct destruction of pathogens, or 
marking of pathogens for destruction by phagocytes
\end{point}

\begin{point}
State that specific antibodies have complementary shapes which fit specific antigens
\end{point}

\begin{point}
Explain that active immunity is gained after an infection by a pathogen, or by vaccination
\end{point}

Each lymphocyte produces only a specific antibody, which has only a specific antigen to which it
can bind. The binding of antibodies to pathogens' antigens destroys the pathogen outright or
these antibodies can mark these pathogens by joining many of them together to be engulfed and
destroyed by phagocytes by phagocytosis.

\begin{point}
Outline the process of vaccination: 
\begin{enumerate}
	\setlength\itemsep{0em}
	\item weakened pathogens or their antigens are given
	\item the antigens stimulate an immune response by lymphocytes which produce antibodies
	\item memory cells are produced that give long-term immunity
\end{enumerate}
\end{point}

\underline{\emph{Memory cells}} are a type of lymphocyte that last for very long in the blood.
These cells are responsible for recognising a certain pathogen. When the pathogen is detected,
these cells divide, increasing in number and produce antibodies to destroy the pathogen.

\begin{point}
Explain the role of vaccination in controlling the spread of transmissible diseases
\end{point}

Vaccination protects those who are vaccinated and those who are not, as the disease against which
the vaccine acts will be left with very few places to replicate (unvaccinated individuals). Hence, 
vaccinating enough people in a population results in \underline{\emph{herd immunity}}, where
unvaccinated and vaccinated people are safe of the disease as a result of what was explained
above.
individuals

\begin{point}
Explain that passive immunity is a short-term defence against a pathogen by antibodies acquired from 
another individual, limited to: across the placenta and in breast milk
\end{point}

Antibodies of the mother are passed onto the fetus and baby, into the blood via placental exchange
and via ingestion of breast milk, respectively. These antibodies last only a short time.

\begin{point}
Explain the importance of breast-feeding for the development of passive immunity in infants
\end{point}

An infant's immune system is not well developed, and these antibodies from its mother can
protect it against any diseases to which the mother is immune.

\begin{point}
State that memory cells are not produced in passive immunity
\end{point}

\begin{point}
Outline how HIV affects the immune system, limited to: decreased lymphocyte numbers and reduced 
ability to produce antibodies, which weakens the immune system
\end{point}

See Section 1.1.8 onwards, if necessary.
